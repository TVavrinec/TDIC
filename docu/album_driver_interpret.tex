\section{Popis databázových terminálových funkcí}
\subsection{Private funkce}

\begin{lstlisting}[style=CStyle]
bool readWold(char *buffer, char *word, int *length);
int getComperFaktor(char *compare_faktor);
bool getComperDirectory(char *compare_faktor);
\end{lstlisting}
Tyto funkce slouží pro zvýšení přehlednosti kódu a jejich deklarace je podle mě plně vystihuje.

\subsection{Public funkce}
\begin{lstlisting}[style=CStyle]
/*
    parse text, loaded file from file path from parameter, and print album list.
    you can use " ." as default file path "output/album_list.csv"
    you can coll the function from terminal as "load-file <file_path>"
*/
void loadFile_i(char *file_path);
\end{lstlisting}
Načte databázi ze souboru na adrese zadané v předaném textu, který nejprve zkontroluje.
Pro pohodlnost můžete využívat výchozí soubor na adrese "output/album\_list.csv" pomocí parametru " .".\\ 
Příklad volání: "load-file output/album\_list.csv" nebo "load-file ."

\begin{lstlisting}[style=CStyle]
/*
    parse text and deleted album with name from the text
    you can coll the function from terminal as "del-album "
*/
void delAlbum_i(char *album_name);
\end{lstlisting}
Vymaže z databáze album, jehož jméno jste zadali\\ 
Příklad volání: "del-album Vlasy" pro smazání alba s názvem Vlasy.

\begin{lstlisting}[style=CStyle]
/*
    parse text and save album list to file on file address in the text 
    the default address is "output/album_list.csv"
    you can coll the function from terminal as "save-album-list <file_path>"
*/
void saveAlbumsList_i(char *file_path);
\end{lstlisting}
Uloží databázi do souboru na zadané adrese.
Opět můžete využít " ." namísto zdlouhavého "output/album\_list.csv".\\ 
Příklad volání: "save-album-list output/album\_list.csv" nebo "save-album-list ."

\begin{lstlisting}[style=CStyle]
/*
    print all album
    you can coll the function from terminal as "list "
*/
void printfAllAlbums_i();
\end{lstlisting}
Vypíše všechna alba v databázi.\\ 
Příklad volání: "list".

\begin{lstlisting}[style=CStyle]
/*
    parse text and make a new album with data from text
    data taken in order:
    name interpreter year genre score
    divided by spaces
    you can call the function from a terminal as "add-album <album data>"
*/
void addNewAlbum_i(char *album); 
\end{lstlisting}
Vytvoří nové album z předaných parametrů.\\ 
Příklad volání: "add-album jmeno interpret 2010 zanr 8" pro vytvoření alba se jménem "jmeno" od autora "interpret" z roku "2010" žánru "zanr" a hodnocením 8.
\newpage
\begin{lstlisting}[style=CStyle]
/*
    parse text and sort album according to "compare_faktor" with the director from "sort_dir".
    compare_factor:
        name
        interpreter
        year
        genre
        score
    sort_dir: defoult is ascending
        VV
    to end, print the album list
    you can call the function from a terminal as "sort-albums <compare_faktor> <sort_dir>"
*/
void sortAlbums_i(char *compare_factor_and_sort_dir);
\end{lstlisting}
Seřadí alba podle zadaného parametru, ve výchozím stavu vzestupně a při zadání přepínače "VV" sestupně.
Parametr lze zadat klíčovými slovy "name", "interpreter", "year", "genre" a "score".\\ 
Příklad volání: "sort-albums score" pro seřazení alb podle jejich hodnocení.

\begin{lstlisting}[style=CStyle]
/*
    print album count
    you can call the function from a terminal as "album-count "
*/
void getAlbumCount_i();
\end{lstlisting}
Spočítá alba v databázi a vypíše jejich počet.\\ 
Příklad volání: "album-count".

\begin{lstlisting}[style=CStyle]
/*
    parse text and print album
    you can call the function from a terminal as "album <name>"
*/
void getAlbum_i(char *name);
\end{lstlisting}
Vypíše záznam o albu zadaného jména.
Předpokládá, že jméno slouží jako jednoznačný identifikátor alba.\\ 
Příklad volání: "album Vlasy" pro výpis záznamu o albu Vlasy.

\begin{lstlisting}[style=CStyle]
/*
    parse text and filter album list.
    the argument is desired values in filtered list in order:
    name interpreter year genre score
    divided by spaces
    if some value is not important you can replace it with "-" symbol.

    for example, you can use the command "filter-album - - - Rock -" to filter Rock albums
    you can call the function from a terminal as "filter-album <filter seting>"
*/
void getAlbumSortedList_i(char *album_prototype);
\end{lstlisting}
Vyfiltruje alba, jejichž položky odpovídají zadaným.
Pokud je třeba filtrovat jen podle některých položek, zadejte místo položek nevyužitých "-".
Všechny parametry musí být vždy odděleny mezerou.\\ 
Příklad volání: "filter-album - - 2003 - -" pro vyfiltrovat všech alb z roku 2003.

\begin{lstlisting}[style=CStyle]
/*
    parse text and save filtered album list to file on address in the text.
    the default address is "output/filtered_list.csv"
    you can call the function from a terminal as "save-filter <file_path>"
*/
void saveFilteredAlbumList_i(char *file_path);
\end{lstlisting}
Uloží vyfiltrovaná alba do souboru, který je předán v parametru.
Opět lze použít "." namísto "output/filtered\_list.csv" jakožto výchozí soubor.\\ 
Příklad volání: "save-filter output/filtered\_list.csv" nebo "save-filter .".
\newpage
\begin{lstlisting}[style=CStyle]
/*
    parse text and album with entered name.
    the argument is desired values in filtered list in order:
    name interpreter year genre score
    divided by spaces
    if some value is not change you replace it with "-" symbol.
    
    for example, you can use the command "change-album Vlasy - 2020 - -" to change yer of publication in album "Vlasy"
    you can call the function from a terminal as "change-album <name> <changing param>"
*/
void changeAlbumRecord(char *change_list);
\end{lstlisting}
Přepíše kteroukoli položku alba krom jména alba, podle kterého identifikuje upravované album.
Parametry, které mají zůstat stejné, budou nahrazeny pomlčkou "-".\\ 
Příklad volání: "change-album Vlasy - - - 8" pro změnu původního hodnocení alba "Vlasy" na hodnocení 8. 

\begin{lstlisting}[style=CStyle]
/*
    print all albums in the filtered list
    you can coll the function from terminal as "print-filter "
*/
void printFilteredAlbumList_i();
\end{lstlisting}
Vypíše všechna alba, která která jsou aktuálně vyfiltrovaná.\\ 
Příklad volání: "print-filter ". 