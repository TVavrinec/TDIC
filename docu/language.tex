
\section{Parser} 
\subsection{Private funkce aritmetické části parseru}
\begin{lstlisting}[style=CStyle]
bool isNumber(char x);     
bool isOperator(char x);   
bool isMathSim(char x);    
bool isLatter(char x);     
bool isSymbolValid(char x);
\end{lstlisting}
Tyto funkce slouží pro pohodlnější definici podmínek a jediné co dělají je, že ověřují, zda je daný znak číslo, operátor atd.

\begin{lstlisting}[style=CStyle]
readed_number_t loadAnother(char *another);
readed_number_t loadNumber(char *number);  
readed_number_t loadBracket(char *bracket);
\end{lstlisting}
Tyto funkce slouží matematickému parseru pro přečtení části řetězce s matematickým výrazem.
Funkce \("loadAnother"\) slouží jako rozcestí, které rozhoduje, zda se má zavolat \(loadNumber\) nebo \(laadBracket\).
Všechny tři následně vracejí strukturu, která obsahuje hodnotu dané části výpočtu a její délku, kterou zabírala ve vstupním řetězci.
\\
\begin{lstlisting}[style=CStyle]
readed_number_t combineNumber(readed_number_t a, readed_number_t b, char marker);
\end{lstlisting}
Tato funkce slouží pro samotné kombinování (sčítání, odčítání atd.) čísel ze struktur "a" a "b" podle znaménka v parametru "marker".
\\

\subsection{Private funkce databázového části parseru}
\begin{lstlisting}[style=CStyle]
void doDatabaseFunction(char *command, int command_length);
\end{lstlisting}
Tato funkce je rozcestí pro konkrétní databázové funkce, které jsou vždy popsané při své deklaraci.

\subsection{Funkce pro použití mimo knihovnu}
\begin{lstlisting}[style=CStyle]
/*
    load number from text and return his value and his length in text format
*/
readed_number_t loadNumber(char *number);
\end{lstlisting}
Načte číslo z textu v paměti na adrese "number" a vrátí ho ve struktuře "readed\_number\_t".
Tato struktura obsahuje číselnou hodnotu čteného čísla a počet znaků, který v textu zaujímalo.

\begin{lstlisting}[style=CStyle]
/*
    load line from the terminal and add ";;" to the end of the string
*/
int readLine(char *buffer);
\end{lstlisting}
Načte do paměti na adresu "buffer" textu, který uživatel zadal do konzole a ukončí jej dvěma středníky ";;" podle kterých se pak orientuje následující část programu. 

\begin{lstlisting}[style=CStyle]
/*
    interpret text ended ";;"
    you can calculate math expressions and do database functions
*/
bool interpretLine(char *line);
\end{lstlisting}
Spočítá matematické příklady a vykoná databázové příkazy uložené v textovém řetězci v paměti na adrese "line".
Předpokládá zakončení řetězce dvěma středníky ";;".

\begin{lstlisting}[style=CStyle]
/*
    calculate expressions from text ended ";;"
*/
readed_number_t interpretArithmeticsFunction(char *example);
\end{lstlisting}
Vypočte matematický příklad uložený v textovém řetězci v paměti na adrese "example".
Předpokládá zakončení řetězce dvěma středníky ";;".

\begin{lstlisting}[style=CStyle]
/*
    do databas function describ in text "command" and end ";;"
    function is describ in file "album_driver_interpret.h"
*/
void interpretDatabaseFunction(char *command);
\end{lstlisting}
vykoná databázové příkazy uložené v textovém řetězci v paměti na adrese "command".
Předpokládá zakončení řetězce dvěma středníky ";;".